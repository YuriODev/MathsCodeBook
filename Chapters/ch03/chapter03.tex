\chapter{Linear Algebra and Matrices}

\section{Matrix Operations}

\subsection{Definition: Matrices}
\begin{defi}[Matrix Definition]
A matrix is a rectangular array of numbers arranged in rows and columns. The size or dimension of a matrix is given by the number of rows and columns. A matrix with $m$ rows and $n$ columns is called an $m \times n$ matrix.
\end{defi}

\subsection{Theorem: Properties of Matrix Operations}
\begin{theorem}[Matrix Addition and Multiplication]
Let $A$ and $B$ be $m \times n$ matrices, and $C$ an $n \times p$ matrix. The following properties hold:
\begin{itemize}
    \item Matrix Addition: $(A + B)_{ij} = A_{ij} + B_{ij}$
    \item Scalar Multiplication: $(cA)_{ij} = c \cdot A_{ij}$
    \item Matrix Multiplication: $(AB)_{ij} = \sum_{k=1}^{n} A_{ik}B_{kj}$
\end{itemize}
\end{theorem}

\subsection{Example: Matrix Multiplication}
\begin{example}[Matrix Multiplication]
Let $A = \begin{bmatrix} 1 & 2 \\ 3 & 4 \end{bmatrix}$ and $B = \begin{bmatrix} 2 & 0 \\ 1 & 2 \end{bmatrix}$. The product $AB$ is calculated as:
\[
AB = \begin{bmatrix} 1 & 2 \\ 3 & 4 \end{bmatrix} \begin{bmatrix} 2 & 0 \\ 1 & 2 \end{bmatrix} = \begin{bmatrix} 1(2) + 2(1) & 1(0) + 2(2) \\ 3(2) + 4(1) & 3(0) + 4(2) \end{bmatrix} = \begin{bmatrix} 4 & 4 \\ 10 & 8 \end{bmatrix}
\]
\end{example}

\newpage
\section{Determinants}

\subsection{Theorem: Properties of Determinants}
\begin{theorem}[Determinant Properties]
Let $A$ and $B$ be square matrices. The determinant satisfies the following properties:
\begin{itemize}
    \item $\det(A) = \det(A^T)$ (Determinant of transpose is the same as the determinant)
    \item $\det(AB) = \det(A)\det(B)$ (Multiplicative property)
    \item If $A$ is triangular, then $\det(A)$ is the product of its diagonal entries
\end{itemize}
\end{theorem}

\subsection{Example: Determinant of a 3x3 Matrix}
\begin{example}[Determinant Calculation]
Let $A = \begin{bmatrix} 1 & 2 & 3 \\ 0 & 1 & 4 \\ 5 & 6 & 0 \end{bmatrix}$. The determinant of $A$ is:
\[
\det(A) = 1 \cdot \det\begin{bmatrix} 1 & 4 \\ 6 & 0 \end{bmatrix} - 2 \cdot \det\begin{bmatrix} 0 & 4 \\ 5 & 0 \end{bmatrix} + 3 \cdot \det\begin{bmatrix} 0 & 1 \\ 5 & 6 \end{bmatrix}
\]
After simplifying:
\[
\det(A) = 1(1 \cdot 0 - 4 \cdot 6) - 2(0 \cdot 0 - 5 \cdot 4) + 3(0 \cdot 6 - 5 \cdot 1) = -24 + 40 - 15 = 1
\]
\end{example}


\section{Inverse of a Matrix}

\subsection{Theorem: Inverse of a Matrix}
\begin{theorem}[Invertibility Condition]
A square matrix $A$ is invertible if and only if $\det(A) \neq 0$. The inverse of $A$ is denoted by $A^{-1}$ and satisfies the property $AA^{-1} = A^{-1}A = I$, where $I$ is the identity matrix.
\end{theorem}

\subsection{Example: Inverse of a 2x2 Matrix}
\begin{example}[Matrix Inverse]
Let $A = \begin{bmatrix} 4 & 7 \\ 2 & 6 \end{bmatrix}$. The inverse of $A$ is calculated using the formula:
\[
A^{-1} = \frac{1}{\det(A)} \begin{bmatrix} d & -b \\ -c & a \end{bmatrix}
\]
where $\det(A) = ad - bc$. For this matrix:
\[
\det(A) = 4(6) - 7(2) = 10, \quad A^{-1} = \frac{1}{10} \begin{bmatrix} 6 & -7 \\ -2 & 4 \end{bmatrix} = \begin{bmatrix} 0.6 & -0.7 \\ -0.2 & 0.4 \end{bmatrix}
\]
\end{example}

\subsection{Python Code Snippet: Matrix Inversion}
\begin{codesnippet}[Matrix Inversion in Python]
\custominputminted{python}{Chapters/ch03/matrix_inverse.py}
\end{codesnippet}

\section{Eigenvalues and Eigenvectors}

\subsection{Definition: Eigenvalues and Eigenvectors}
\begin{defi}[Eigenvalue Definition]
Let $A$ be an $n \times n$ matrix. A non-zero vector $v$ is called an eigenvector of $A$ if there exists a scalar $\lambda$, called the eigenvalue, such that $Av = \lambda v$.
\end{defi}

\subsection{Example: Eigenvalues of a 2x2 Matrix}
\begin{example}[Eigenvalue Calculation]
Let $A = \begin{bmatrix} 4 & 1 \\ 2 & 3 \end{bmatrix}$. The eigenvalues of $A$ are found by solving the characteristic equation:
\[
\det(A - \lambda I) = 0
\]
For this matrix, the characteristic equation is:
\[
\det\begin{bmatrix} 4 - \lambda & 1 \\ 2 & 3 - \lambda \end{bmatrix} = (4 - \lambda)(3 - \lambda) - 2 = \lambda^2 - 7\lambda + 10 = 0
\]
The eigenvalues are $\lambda_1 = 5$ and $\lambda_2 = 2$.
\end{example}

\subsection{Python Code Snippet: Eigenvalue Calculation}
\begin{codesnippet}[Eigenvalue and Eigenvector Calculation in Python]
\custominputminted{python}{Chapters/ch03/eigenvalues.py}
\end{codesnippet}

\section{Gaussian Elimination}

\subsection{Theorem: Gaussian Elimination}
\begin{theorem}[Gaussian Elimination]
Gaussian elimination is a method for solving a system of linear equations. It works by transforming the system's augmented matrix into row echelon form, and then solving the system using back-substitution.
\end{theorem}

\subsection{Example: Solving a System of Equations}
\begin{example}[Gaussian Elimination]
Solve the following system of equations using Gaussian elimination:
\[
\begin{aligned}
    2x + 3y - z &= 1 \\
    4x + y + 2z &= 2 \\
    -2x + 7y + 3z &= 3
\end{aligned}
\]
The augmented matrix is:
\[
\begin{bmatrix}
2 & 3 & -1 & | & 1 \\
4 & 1 & 2 & | & 2 \\
-2 & 7 & 3 & | & 3
\end{bmatrix}
\]
Apply row operations to reduce the matrix to row echelon form, and then solve the system using back-substitution.
\end{example}

\subsection{Python Code Snippet: Gaussian Elimination}
\begin{codesnippet}[Gaussian Elimination in Python]
\custominputminted{python}{Chapters/ch03/gaussian_elimination.py}
\end{codesnippet}